\chapter{PENUTUP}
\label{chap:penutup}

% Ubah bagian-bagian berikut dengan isi dari penutup

\section{Kesimpulan}
\label{sec:kesimpulan}

\par Berdasarkan pengujian yang dilakukan pada penelitian ini, ditarik beberapa kesimpulan sebagai berikut:

\begin{enumerate}[nolistsep]
    % \item Bobot atau \emph{weight} yang dilatih menggunakan \emph{Pretrained Weights} dari YOLOv5 dengan \emph{weights} yang dilatih hanya menggunakan dataset Deteksi Helm Keselamatan Kerja tidak memiliki perbedaan signifikan dari segi \emph{precision}, \emph{recall}, \emph{mAP}, maupun \emph{inference speed} nya untuk setiap varian (N,S,M,L) saat dilakukan pengujian menggunakan test set dari dataset Deteksi Helm Keselamatan Kerja dimana pada metriknya dimana untuk rata - rata \emph{precision} mendapatkan nilai 0,92 ,  \emph{recall}  0,87 dan mAP@.5 0,92.
    \item YOLOv5 dapat digunakan untuk deteksi helm keselamatan kerja dibuktikan pada pengujian masing - masing model dengan \emph{Pretrained Weights} dari YOLOv5 ataupun tanpa \emph{Pretrained Weights} untuk setiap varian (N,S,M,L) dimana mendapatkan rata - rata \emph{precision} 0,92, \emph{recall}  0,87 dan mAP@.5 0,92 yang dimana juga tidak ada perbedaan signfikan diantara varian.
    % \item Pengujian dengan variasi jarak dari 1,3 meter hingga 9 meter memiliki nilai rata-rata untuk \emph{precision} 0,9, \emph{recall} 0,97 dan mAP@.5 0,98.

    \item Pada pengujian kecepatan \emph{inference}, jika diurutkan dari paling cepat hingga yang paling lambat yaitu \emph{Nano}(N), \emph{Small}(S), \emph{Medium}(M), lalu \emph{Large}(L) dimana juga berpengaruh pada \emph{frame-rate} yang dibuktikan pada pengujian di Jetson Nano dimana varian N mendapatkan 18,4 FPShingga varian L mendapatkan 1,8 FPS

    \item Sistem deteksi helm keselamatan dapat dilakukan pada pengujian jarak 1 meter hingga 10 meter dibuktikan melalui pengujian model mendapatkan nilai rata-rata pada semua jarak untuk \emph{precision} 0,9, \emph{recall} 0,97 dan mAP@.5 0,98  dan juga dari pengujian sistem alarm mendapatkan akurasi paling rendah 0.92.
    % \item Perbedaan performa yang signifikan dari \emph{weight} yang dilatih menggunakan \emph{Pretrained Weights} dari YOLOv5 dengan \emph{weights} yang dilatih hanya menggunakan dataset Deteksi Helm Keselamatan Kerja baru terlihat pada pengujian pada keadaan pencahayaan rendah dimana \emph{precision} pada \emph{weight} yang dilatih menggunakan \emph{Pretrained Weights} dari YOLOv5 lebih baik dari pada yang murni menggunakan dataset Deteksi Helm Keselamatan Kerja. Pada nilai \emph{recall} secara umum tidak ada berbedaan signifikan pada tiap varian tetapi khusus untuk varian "hedec\textunderscore pretrain\textunderscore N" memiliki nilai \emph{recall} yang sangat buruk yaitu 0.373 karena gagal untuk mendeteksi kelas “no\textunderscore helmet”.
    \item Sistem deteksi helm keselamatan kerja dapat dilakukan pada pencahayaan rendah dibuktikan melalui pengujian model dengan nilai rata-rata dari semua varian model untuk \emph{precision} 0,76, \emph{recall} 0,74 dan mAP@.5 0,78 dan juga dari pengujian sistem alarm dengan varian \emph{Small} mendapatkan nilai 0,82 dimana menunjukkan terjadi penurunan performa pada pencahayaan rendah.
    % \item Dari varian \emph{weight} yang dilatih menggunakan model YOLOv5 yaitu \emph{Nano}(N), \emph{Small}(S), \emph{Medium}(L), \emph{Large}(L), nilai metrik untuk \emph{precision},\emph{recall}, dan mAP mengalami peningkatan sedangkan \emph{inference speed} menjadi lebih lama jika diurutkan dari \emph{Nano} hingga \emph{Large}. Kecuali pada pengujain di pencahayaan rendah dimana varian \emph{Large} mengalami penurunan walaupun \emph{inference speed} nya tetap lebih lama.
    \item Sistem deteksi helm keselamatan kerja dapat dijalankan secara \emph{real-time} ditunjukkan pada pengujian sistem di Jetson-Nano dengan akurasi 0.92.
    % \item Pada pengujian performa Deteksi Helm Keselamatan Kerja menggunakan YOLOv5 di Jetson Nano, semua varian kecuali \emph{Large} berhasil dijalankan dengan \emph{Frame Rate} 15 FPS hingga 24 FPS. Untuk setiap varian \emph{Large} selalu gagal untuk dijalankan.

    \item Menimbang hasil pengujian dari semua model yang tidak terlalu signifikan dengan kecepatan \emph{inference}-nya, varian model YOLOv5s dengan \emph{weight} dengan \emph{Pretrained Weight} varian \emph{Small} dianggap sebagai pilihan paling optimal pada keperluan deteksi helm keselamatan kerja secara \emph{real-time} yang lalu juga digunakan untuk semua pengujian sistem.
    
\end{enumerate}

\section{Saran}
\label{chap:saran}

\par Untuk keperluan pengembangan dari penelitian ini, terdapat beberapa saran yang dapat diambil yaitu:

\begin{enumerate}[nolistsep]

  \item Menambahkan dataset untuk keperluan train, terutama pada keadaan pencahayaan rendah dan untuk kelas \emph{no\textunderscore helmet}.
  

  \item Membuat sistem pemicu alarm yang lebih tolerir jika terdeteksi ada yang tidak menggunakan helm keselamatan kerja sepersekian detik.
  
  \item Mengembangkan fungsi pemicu alarm yang lebih modular sehingga bisa menggunakan alarm jenis lain selain alarm suara.

  \item Membuat tampilan atau \emph{user interface} yang lebih mudah dipahami.

  \item Menambahkan \emph{head gear} lainnya pada dataset yang kemungkinan akan digunakan oleh pekerja konstruksi yang tidak termasuk helm keselamatan kerja seperti topi dan shemag atau balaclava.
  \item Melakukan pengujian pada berbagai varian webcam atau kamera CCTV.

\end{enumerate}
