\chapter{PENUTUP}
\label{chap:penutup}

% Ubah bagian-bagian berikut dengan isi dari penutup

\section{Kesimpulan}
\label{sec:kesimpulan}

\par Berdasarkan pengujian yang dilakukan pada penelitian ini, ditarik beberapa kesimpulan sebagai berikut:

\begin{enumerate}[nolistsep]
    \item Bobot atau \emph{weight} yang dilatih menggunakan \emph{Pretrained Weights} dari YOLOv5 dengan \emph{weights} yang dilatih hanya menggunakan dataset Deteksi Helm Keselamatan Kerja tidak memiliki perbedaan signifikan dari segi \emph{precision}, \emph{recall}, \emph{mAP}, maupun \emph{inference speed} nya untuk setiap varian (N,S,M,L) saat dilakukan pengujian menggunakan test set dari dataset Deteksi Helm Keselamatan Kerja dimana pada metriknya dimana untuk rata - rata \emph{precision} mendapatkan nilai 0,92 ,  \emph{recall} diatas 0,87 dan mAP@.5 0,92.
    \item Pengujian dengan variasi jarak dari 1,3 meter hingga 9 meter memiliki nilai rata-rata untuk \emph{precision} 0,9, \emph{recall} 0,97 dan mAP@.5 0,98.
    \item Perbedaan performa yang signifikan dari \emph{weight} yang dilatih menggunakan \emph{Pretrained Weights} dari YOLOv5 dengan \emph{weights} yang dilatih hanya menggunakan dataset Deteksi Helm Keselamatan Kerja baru terlihat pada pengujian pada keadaan pencahayaan rendah dimana \emph{precision} pada \emph{weight} yang dilatih menggunakan \emph{Pretrained Weights} dari YOLOv5 lebih baik dari pada yang murni menggunakan dataset Deteksi Helm Keselamatan Kerja. Pada nilai \emph{recall} secara umum tidak ada berbedaan signifikan pada tiap varian tetapi khusus untuk varian "hedec\textunderscore pretrain\textunderscore N" memiliki nilai \emph{recall} yang sangat buruk yaitu 0.373 karena gagal untuk mendeteksi kelas “no\textunderscore helmet”.
    \item Dari varian \emph{weight} yang dilatih menggunakan model YOLOv5 yaitu \emph{Nano}(N), \emph{Small}(S), \emph{Medium}(L), \emph{Large}(L), nilai metrik untuk \emph{precision},\emph{recall}, dan mAP mengalami peningkatan sedangkan \emph{inference speed} menjadi lebih lama jika diurutkan dari \emph{Nano} hingga \emph{Large}. Kecuali pada pengujain di pencahayaan rendah dimana varian \emph{Large} mengalami penurunan walaupun \emph{inference speed} nya tetap lebih lama.
    \item Pada pengujian performa Deteksi Helm Keselamatan Kerja menggunakan YOLOv5 di Jetson Nano, semua varian kecuali \emph{Large} berhasil dijalankan dengan \emph{Frame Rate} 15 FPS hingga 24 FPS. Untuk setiap varian \emph{Large} selalu gagal untuk dijalankan.
    \item Pada keadaan pencahayaan cukup dan keperluan deteksi paling cepat, bobot "hedec\textunderscore pretrain\textunderscore N" ,yang merupakan bobot yang dilatih menggunakan \emph{pretrained weight} varian \emph{Nano} dari YOLOv5, sudah sangat mencukupi karena memiliki nilai \emph{precision}, \emph{recall}, dan \emph{mAP} yang tidak jauh berbeda dengan varian lain diatasnya tetapi memiliki \emph{inference speed} yang jauh lebih cepat.
    \item Pada keadaan pencahayaan rendah, bobot "hedec\textunderscore pretrain\textunderscore M" ,yang merupakan bobot yang dilatih menggunakan \emph{pretrained weight} varian \emph{Nano} dari YOLOv5, memiliki nilai \emph{precision}, \emph{recall}, dan \emph{mAP} paling optimal dibanding varian lainnya.
    \item Sistem \emph{alarm trigger} berhasil berjalan ketika terdapat kelas \emph{no\textunderscore helmet} yang terdeteksi pada proses \emph{inference}.
    
\end{enumerate}

\section{Saran}
\label{chap:saran}

\par Untuk keperluan pengembangan dari penelitian ini, terdapat beberapa saran yang dapat diambil yaitu:

\begin{enumerate}[nolistsep]

  \item Menambahkan dataset untuk keperluan train, terutama pada keadaan pencahayaan rendah dan untuk kelas \emph{no\textunderscore helmet} yang memilki hasil pengujian rendah.

  \item Membuat sistem pemicu alarm yang lebih tolerir jika terdeteksi ada yang tidak menggunakan helm keselamatan kerja sepersekian detik.
  
  \item Mengembangkan fungsi pemicu alarm yang lebih modular sehingga bisa menggunakan alarm jenis lain selain alarm suara.

  \item Membuat tampilan atau \emph{user interface} yang lebih mudah dipahami.

  \item Menambahkan \emph{head gear} lainnya yang kemungkinan akan digunakan oleh pekerja konstruksi yang tidak termasuk helm keselamatan kerja seperti topi dan shemag atau balaclava.
  \item Melakukan pengujian deteksi untuk jarak lebih dari 9 meter dan resolusi yang lebih tinggi dari 640p.

\end{enumerate}
