\chapter{HASIL DAN PEMBAHASAN}
\label{chap:hasilpembahasan}

% Ubah bagian-bagian berikut dengan isi dari hasil dan pembahasan

Bab ini akan membahas hasil dan analisa dari desain sistem yang sudah dibuat dan implementasinya. Pengujian terhadap hasil dibagi menjadi beberap bagian.

\section{Pengujian Performa antar \emph{Weight}}
\label{sec:ujiperforma}
\par Repositori YOLOv5 menyediakan beberapa \emph{checkpoint} atau \emph{weight} yang 
merupakan hasil training model YOLOv5 dari dataset COCO yang dimanfaatkan sebagai 
\emph{pretrained model}. Ekspektasi penggunaan pretrained model ini yaitu bobot yang 
dihasilkan akan memiliki performa yang lebih tinggi daripada melakukan training tanpa 
pretrained model sama sekali. \emph{COCO Dataset} ini memiliki 80 \emph{class} berbeda. 
Seperti disebutkan pada \ref{sec:pelabelandataset} yaitu untuk keperluan penelitian ini 
hanya memerlukan 2 kelas yaitu "no\textunderscore helmet" dan "with\textunderscore helmet". 
Pada bagian ini akan dipaparkan dan dibandingkan performa antara tiap bobot yang dihasilkan 
dengan model pretrained dan yang tidak menggunakan pretrained model. Pengujian ini dilakukan menggunakan
\emph{resource} dari Google Colab.

\subsection{Pengujian Performa \emph{Weight} dari Hasil \emph{Pretrain} COCO Dataset}
\label{subsec:ujiperforma_coco}
Beberapa \emph{pretrained model} yang disedikana dari repositori YOLOv5 yaitu YOLOv5n, YOLOV5s, YOLOv5m, YOLOv5l,YOLOv5l, dan YOLOv5x. Selain beberapa model pretrained tersebut juga ada versi untuk ukuran gammbar 1280 yaitu seri YOLOv5n6 hingga YOLOv5l6. Pada penelitian ini hanya menggunakan variasi YOLOv5n hingga YOLOv5l karena variasi YOLOv5x dan versi seri 6 untuk ukuran gambar 1280 membutuhkan waktu yang lebih lama. Perbedaan - perbedaan yang ada pada bobot - bobot pretrained tersebut berasal dari konfigurasi awal dari training pada dataset COCO yang menggunakan YOLOv5, terutama pada paramter "depth\textunderscore multiple" dan "width\textunderscore multiple".
Proses validasi hasil training dilakukan menggunakan dataset Deteksi Helm Keselamatan Kerja yang pembagiannya dijelaskan pada bagian \ref{sec:preprocessing} yang berjumlah 1200 gambar dengan kelas "no\textunderscore helmet" berjumlah 1322 label dan kelas "with\textunderscore helmet" berjumlah 4294 label. Dari semua \emph{weight} yang dihasilkan dari training menggunkana Dataset Helm Keselamatan Kerja lalu dilakukan validasi yang hasilnya dapat dilihat pada Tabel~\ref{tb:pretraincoco}.


\begin{longtable}{|c|c|c|c|c|}
  \caption{Hasil Validasi \emph{Weight} dari Hasil Pretrain}
  \label{tb:pretraincoco}\\
  \hline
  \rowcolor[HTML]{C0C0C0}
  \textbf{Nama Bobot} & \emph{Precision}  & \emph{Recall} & \emph{mAP@.5} & \emph{Inference Time}\\
  \hline
  hedec\textunderscore pretrain\textunderscore N & 0.92               & 0.878         & 0.922         & 1.9ms              \\
  hedec\textunderscore pretrain\textunderscore S & 0.927              & 0.882         & 0.929         & 4.1ms              \\
  hedec\textunderscore pretrain\textunderscore M & 0.923              & 0.892         & 0.933        & 9ms                  \\
  hedec\textunderscore pretrain\textunderscore L & 0.917              & 0.87          & 0.919         & 13.9ms                \\
  \hline
\end{longtable}

Berdasarkan hasil validasi, tidak ada perbedaan signifikan dari \emph{precision, recall, mAP}. Tetapi perbedaan besar ada pada \emph{inference time}-nya. 

\subsection{Pengujian Performa \emph{Weight} Hasil Train Murni Dataset Deteksi Helm Keselamatan Kerja}
\label{subsec:murnidataset}

Pada bagian ini akan dipaparkan hasil validasi dari \emph{weight} yang dihasilkan dari train tanpa menggunakan \emph{pretrained weight} yang disediakan dari repositori YOLOv5. Untuk proses training, akan digunakan konfigurasi yang serupa dengan konfigurasi yang digunakan pada \emph{weight} yang disediakan repositori YOLOv5.

\begin{longtable}{|c|c|c|c|c|}
  \caption{hasil Validasi \emph{Weight} dari Murni Dataset}
  \label{tb:nopretrain}\\
  \hline
  \rowcolor[HTML]{C0C0C0}
  \textbf{Nama Bobot} & \emph{Precision}  & \emph{Recall} & \emph{mAP@.5} & \emph{Inference Time}\\
  \hline
  hedec\textunderscore pure\textunderscore N  & 0.918              & 0.85         & 0.909         & 2.4ms                \\
  hedec\textunderscore pure\textunderscore S  & 0.927             & 0.865        & 0.919        & 4.1ms               \\
  hedec\textunderscore pure\textunderscore M & 0.939            & 0.862        & 0.924        & 8.7ms                \\
  hedec\textunderscore pure\textunderscore L & 0.922            & 0.876      &   0.923       &  14.0ms        \\
  \hline
\end{longtable}

Dari \emph{precision, recall, mAP} tidak berbeda jauh dengan yang menggunakan pretrained checkpoint begitu juga dari \emph{inference time} nya.

\section{Pengujian Performa Berdasarkan Jarak}
\label{sec:ujiberdasarkanjarak}

Pada bagian ini akan memaparkan hasil deteksi pada dataset validasi yang dibagi menjadi beberapa variasi jarak dari kamera. Dataset yang digunakan meliputi 8 foto untuk masing - masing jarak.

\subsection{Pengujian Jarak dengan \emph{Pretrained Weight}}
\label{subsec:ujijarak_pretrainedweight}

\par Bagian ini memaparkan pengujian menggunakan bobot hasil \emph{training} menggunakan \emph{Pretrained Weights}
dari repo YOLOv5 yang selanjutnya disebut sebagai "hedec\_pretrain". 

\begin{enumerate}
  \item \textbf{hedec\_pretrain\_N}
  \item \textbf{hedec\_pretrain\_S}
  \item \textbf{hedec\_pretrain\_M}
  \item \textbf{hedec\_pretrain\_L}
  
  \par Pada pengujian menggunakan bobot "hedec\_pretrain\_L" pada perbedaan jarak, bobot ini memiliki hasil pengujian
  paling baik diantara bobot dimana untuk \emph{precision, recall,} dan \emph{mAP} berada di atas 0.9 untuk semua jarak
  dan kelas. Hasil validasi dapat dilihat pada Tabel~\ref{tb:hasiljarak_hedec_pretrain_L}.

  \par Seperti yang bisa dilihat dari Gambar~\ref{fig:grafvaljarak_hedec_pretrain_L}, hanya terjadi sedikit penurunan presisi
  pada jarak 1.3 meter untuk kelas "with\_helmet". Selain itu semua konstan di atas angka 0.98.

  \begin{table}
    \centering
    \caption{Hasil Validasi Perbedaan Jarak Pada \textbf{"hedec\_pretrain\_L"}}
    \label{tb:hasiljarak_hedec_pretrain_L}
    \begin{tabular}{|l|l|l|l|l|} 
    \hline
    Jarak     & class        & precision & recall & mAP    \\ 
    \hline
    1.3 meter & all          & 0.984     & 1      & 0.995  \\
    2.6 meter & all          & 0.987     & 1      & 0.995  \\
    4 meter   & all          & 0.988     & 1      & 0.995  \\
    5.3 meter & all          & 0.988     & 1      & 0.995  \\
    6.7 meter & all          & 0.989     & 1      & 0.995  \\
    9 meter   & all          & 0.988     & 1      & 0.995  \\
    1.3 meter & no\_helmet   & 0.992     & 1      & 0.995  \\
    2.6 meter & no\_helmet   & 0.99      & 1      & 0.995  \\
    4 meter   & no\_helmet   & 0.989     & 1      & 0.995  \\
    5.3 meter & no\_helmet   & 0.99      & 1      & 0.995  \\
    6.7 meter & no\_helmet   & 0.99      & 1      & 0.995  \\
    9 meter   & no\_helmet   & 0.987     & 1      & 0.995  \\
    1.3 meter & with\_helmet & 0.975     & 1      & 0.995  \\
    2.6 meter & with\_helmet & 0.985     & 1      & 0.995  \\
    4 meter   & with\_helmet & 0.988     & 1      & 0.995  \\
    5.3 meter & with\_helmet & 0.985     & 1      & 0.995  \\
    6.7 meter & with\_helmet & 0.988     & 1      & 0.995  \\
    9 meter   & with\_helmet & 0.988     & 1      & 0.995  \\
    \hline
    \end{tabular}
  \end{table}

  \begin{figure}[ht]
    \centering
    \includegraphics[width=1\textwidth]{gambar/BerdasarkanJarak/hedec_pretrain_L.png}
    \caption{Grafik \emph{Precision, Recall, mAP} untuk \textbf{"hedec\_pretrain\_L"} Pada Jarak 1.3 meter Hingga 9 meter}
    \label{fig:grafvaljarak_hedec_pretrain_L}  
  \end{figure}

\end{enumerate}

\subsection{Pengujian Jarak dengan \emph{Pure Weight}}
\label{subsec:ujijarak_pureweight}

\par Bagian ini memaparkan pengujian menggunakan bobot hasil \emph{training} tanpa menggunakan \emph{Pretrained Weights}
dari repo YOLOv5 yang selanjutnya disebut sebagai "hedec\_pure". 

\begin{enumerate}
  \item \textbf{hedec\_pure\_N}
  \item \textbf{hedec\_pure\_S}
  \item \textbf{hedec\_pure\_M}
  \item \textbf{hedec\_pure\_L}
\end{enumerate}


\newpage

\section{Pengujian Pada Tingkat Kecerahan Rendah}
\label{sec:pengujianberdasarkantingkatkeceharan}

\par Bagian ini memaparkan performa model melakukan
deteksi pada input gambar dengan tingkat kecerahan rendah. Data validasi berisi
35 gambar dengan jumlah label \emph{no\textunderscore helmet} 20 dan label \emph{with\textunderscore helmet} 57.
Validasi dilakukan pada bobot hasil training yang menggunakan \textit{pretrained weight} dan yang tanpa menggunakan \textit{pretrained weight}. 

\subsection{Pengujian Pada Tingkat Kecerahan Rendah dengan \emph{Pretrained Weight}}
\label{subsec:lowlight_pretrained}

\par Berikut merupakan pemaparan hasil validasi pada data validasi menggunakan \emph{weight} yang di-\emph{train} menggunakan
\emph{pretrained weights} yang disedikan repo YOLOv5 yang selanjutnya disebut sebagai "hedec\textunderscore pretain". 
Seperti yang dijelaskan pada Subbab~\ref{subsec:ujiperforma_coco}. 


\begin{enumerate}
  \item \textbf{hedec\textunderscore pretrain\textunderscore N}
  
  \par Dilakukan pengujian kecerahan rendah dengan menggunakan bobot yang di-\emph{train} menggunakan bobot
  pretrain COCO untuk varian \emph{nano} yang merupakan varian paling kecil dari semua bobot yang disediakan 
  dari repo. Didapatkan rata - rata presisi untuk semua kelas 0.767 dan \emph{recall} untuk semua kelas 0.373.
  \par Didapati untuk kelas \emph{no\textunderscore helmet} mendapatkan nilai \emph{precision} 1 dan \emph{recall}
  0. Hal ini dikarenakan objek kepala tanpa menggunakan helm tidak ada yang terdektsi dan atau terdeteksi
  sebagai kelas \emph{with\textunderscore helmet}. 
  
  \begin{longtable}{|c|c|c|c|}
    \caption{Hasil Validasi Pada Tingkat Kecerahan Rendah dengan hedec\textunderscore pretrain\textunderscore N   }
    \label{tb:validasitingkatacerahrendah_yolo5n}\\
    \hline
    % \rowcolor[HTML]{C0C0C0}
    \textbf{\emph{Class} }                     & \textbf{\emph{Precision}}  & \textbf{\emph{Recall}} & \textbf{\emph{mAP@.5}}\\
    \hline
    all                                                 & 0.767          & 0.373        & 0.415         \\
    no\textunderscore helmet                            & 1              & 0            & 0.143          \\
    with\textunderscore helmet                          & 0.534          & 0.745        & 0.687         \\
    \hline
  \end{longtable}
  
  \begin{figure}[h]
    \centering
    \includegraphics[scale=0.2]{gambar/train_v2_val/low_ligjt/yolonano/val_batch0_pred.jpg}
    % \includegraphics[scale=0.1]{gambar/train_v2_val/low_ligjt/yolonano/val_batch0_labels.jpg}
    \caption{Hasil Prediksi Pada Keadaan Rendah dengan hedec\textunderscore pretrain\textunderscore N    }
    % \label{fig:labelbaru}  
  \end{figure}

  \newpage
  \item \textbf{hedec\textunderscore pretrain\textunderscore S}
  
  \par Dilakukan pengujian kecerahan rendah dengan menggunakan bobot yang di-\emph{train} menggunakan bobot
  pretrain COCO untuk varian \emph{small}. Didapatkan rata - rata presisi untuk semua kelas 0.775 dan \emph{recall} untuk semua
  kelas 0.799. Terdapat beberapa \emph{False Positive} dalam prediksi yaitu helm motor yang diletakkan diatas
  motor diprediksi sebagai \emph{with\textunderscore helmet} begitu juga.
  
  \begin{longtable}{|c|c|c|c|}
    \caption{Hasil Validasi Pada Tingkat Kecerahan Rendah dengan hedec\textunderscore pretrain\textunderscore S}
    \label{tb:validasitingkatacerahrendah_yolov5s}\\
    \hline
    % \rowcolor[HTML]{C0C0C0}
    \textbf{\emph{Class} }                     & \textbf{\emph{Precision}}  & \textbf{\emph{Recall}} & \textbf{\emph{mAP@.5}}\\
    \hline
    all                                                 & 0.775          & 0.799        & 0.823         \\
    no\textunderscore helmet                            & 0.862           & 0.65        & 0.713          \\
    with\textunderscore helmet                          & 0.689           & 0.947        & 0.933         \\
    \hline
  \end{longtable}
  
  \begin{figure}[ht]
    \centering
    \includegraphics[scale=0.2]{gambar/train_v2_val/low_ligjt/yolosmall/low_light_val_batch0_pred.jpg}
    % \includegraphics[scale=0.1]{gambar/train_v2_val/low_ligjt/yolosmall/lowlight_val_batch0_labels.jpg}
    \caption{Hasil Prediksi Pada Keadaan dengan hedec\textunderscore pretrain\textunderscore S}
    % \label{fig:labelbaru}  
  \end{figure}


  \newpage
  \item \textbf{hedec\textunderscore pretrain\textunderscore M} 
  
  \par Dilakukan pengujian kecerahan rendah dengan menggunakan bobot yang di-\emph{train} menggunakan bobot
  pretrain COCO untuk varian \emph{medium}. Didapatkan rata - rata presisi untuk semua kelas 0.833 dan \emph{recall} untuk semua
  kelas 0.831. Didapati hasil \emph{precision} dan \emph{recall} untuk kelas \emph{no\textunderscore helm} ,yang biasanya
  memiliki nilai yang kurang bagus pada varian bobot nano , mendapatkan nilai 1 dan 0.68 yang dimana lebih bagus. 
  Varian Medium ini merupakan varian yang memiliki hasil validasi paling tinggi untuk \emph{precision} dan \emph{recall} nya diantara semua \emph{weight}
  yang di \emph{train} menggunakan \emph{pretrained weights} dari YOLOv5 bahkan melebihi varian YOLOv5l yang akan dipaparkan di bagian selanjutnya.
  
  \begin{longtable}{|c|c|c|c|}
    \caption{Hasil Validasi Pada Tingkat Kecerahan Rendah dengan hedec\textunderscore pretrain\textunderscore M}
    \label{tb:validasitingkatacerahrendah_yolov5m}\\
    \hline
    % \rowcolor[HTML]{C0C0C0}
    \textbf{\emph{Class} }                     & \textbf{\emph{Precision}}  & \textbf{\emph{Recall}} & \textbf{\emph{mAP@.5}}\\
    \hline
    all                                                 & 0.833          & 0.831       & 0.893         \\
    no\textunderscore helmet                            & 1              & 0.68        & 0.814         \\
    with\textunderscore helmet                          & 0.666          & 0.982       & 0.973         \\
    \hline
  \end{longtable}
  
  \begin{figure}[ht]
    \centering
    \includegraphics[scale=0.2]{gambar/train_v2_val/low_ligjt/yolomedium/val_batch0_pred.jpg}
    % \includegraphics[scale=0.1]{gambar/train_v2_val/low_ligjt/yolomedium/val_batch0_labels.jpg}
    \caption{Hasil Prediksi Pada Keadaan dengan hedec\textunderscore pretrain\textunderscore M}
    % \label{fig:labelbaru}  
  \end{figure}

  \item \textbf{hedec\textunderscore pretrain\textunderscore L}
  
  \par Dilakukan pengujian kecerahan rendah dengan menggunakan bobot yang di-\emph{train} menggunakan bobot
  pretrain COCO untuk varian \emph{large}. Didapatkan rata - rata presisi untuk semua kelas 0.811 dan \emph{recall} untuk semua
  kelas 0.765 dimana lebih kecil dibandingkan varian \emph{medium}. Didapati juga hasil \emph{precision} dan \emph{recall} untuk kelas \emph{no\textunderscore helm} yang mengalami penurunan dibanding
  menggunakan variant \emph{medium} yaitu 0.914 untuk \emph{precision} dan 0.6 untuk \emph{recall}. Selain itu untuk
  \emph{inference time} nya tentu saja mengalami kenaikan dibandingan dengan varian - varian sebelumnya yang lebih kecil.
  
  \begin{longtable}{|c|c|c|c|}
    \caption{Hasil Validasi Pada Tingkat Kecerahan Rendah dengan hedec\textunderscore pretrain\textunderscore L}
    \label{tb:validasitingkatacerahrendah_yolov5l}\\
    \hline
    % \rowcolor[HTML]{C0C0C0}
    \textbf{\emph{Class} }                     & \textbf{\emph{Precision}}  & \textbf{\emph{Recall}} & \textbf{\emph{mAP@.5}}\\
    \hline
    all                                                 & 0.811          & 0.765       & 0.893         \\
    no\textunderscore helmet                            & 0.914          & 0.6         & 0.724         \\
    with\textunderscore helmet                          & 0.708          & 0.93        & 0.931         \\
    \hline
  \end{longtable}
  
  \begin{figure}[ht]
    \centering
    \includegraphics[scale=0.2]{gambar/train_v2_val/low_ligjt/yololarge/val_batch0_pred.jpg}
    % \includegraphics[scale=0.1]{gambar/train_v2_val/low_ligjt/yololarge/val_batch0_labels.jpg}
    \caption{Hasil Prediksi Pada Keadaan dengan hedec\textunderscore pretrain\textunderscore Ll}
    % \label{fig:labelbaru}  
  \end{figure}

\end{enumerate}


\subsection{Pengujian Pada Tingkat Kecerahan Rendah dengan \emph{Weight} Murni Deteksi Helm Keselamatan Kerja}
\label{subsec:lowlight_pure}

\par Berikut merupakan pemaparan hasil validasi menggunakan data validasi pada keadaan minim pencahayaan menggunakan
\emph{weight} yang merupakan hasil \emph{train} dari dataset Deteksi Helm Keselamatan Kerja tanpa menggunakan \emph{pretrained weight}.
Hasil \emph{train} tanpa menggunakan \emph{pretrain weight} ini selanjutkan akan disebut sebagai "hedec\textunderscore pure".

\begin{enumerate}
  \item \textbf{hedec\textunderscore pure\textunderscore N } 
  
  \par Dilakukan pengujian kecerahan rendah dengan menggunakan bobot yang di-\emph{train} tanpa menggunakan bobot
  pretrain COCO tetapi konfigurasi modelnya dibuat serupa dengan konfigurasi YOLOv5n. 
  Didapatkan rata - rata presisi untuk semua kelas 0.659   dan \emph{recall} untuk semuakelas 0.696.
  \par  Sewajarnya varian ini mendapatkan nilai \emph{precision} dan \emph{recall} paling kecl diantara varian lainnya dan juga \emph{inference speed}
  yang paling cepat dibanding varian lainnya pada \emph{weight} murni \emph{tanpa pretrained weight}.
  
  \begin{longtable}{|c|c|c|c|}
    \caption{Hasil Validasi Pada Tingkat Kecerahan Rendah dengan \emph{Hedec Nano}}
    \label{tb:validasitingkatacerahrendah_hedecN}\\
    \hline
    % \rowcolor[HTML]{C0C0C0}
    \textbf{\emph{Class} }                     & \textbf{\emph{Precision}}  & \textbf{\emph{Recall}} & \textbf{\emph{mAP@.5}}\\
    \hline
    all                                                 & 0.659          & 0.696        & 0.731         \\
    no\textunderscore helmet                            & 0.76           & 0.55         & 0.621         \\
    with\textunderscore helmet                          & 0.558          & 0.842        & 0.842         \\
    \hline
  \end{longtable}
  
  \begin{figure}[h]
    \centering
    \includegraphics[scale=0.2]{gambar/train_v2_val/low_ligjt/customNano/val_batch0_pred.jpg}
    % \includegraphics[scale=0.1]{gambar/train_v2_val/low_ligjt/customNano/val_batch0_labels.jpg}
    \caption{Hasil Prediksi Pada Keadaan dengan \emph{Weight Hedec Nano}}
    % \label{fig:labelbaru}  
  \end{figure}
  
  
  \item \textbf{hedec\textunderscore pure\textunderscore S }
  
  \par Dilakukan pengujian kecerahan rendah dengan menggunakan bobot yang di-\emph{train} tanpa menggunakan bobot
  pretrain COCO tetapi konfigurasi modelnya dibuat serupa dengan konfigurasi YOLOv5s. 
  Didapatkan rata - rata presisi untuk semua kelas 0.706   dan \emph{recall} untuk semuakelas 0.696 dimana sedikit lebih baik
  dibandingkan varian \emph{Nano} sebelumnya.
  
  \begin{longtable}{|c|c|c|c|}
    \caption{Hasil Validasi Pada Tingkat Kecerahan Rendah dengan \emph{Hedec Small}}
    \label{tb:validasitingkatacerahrendah_hedecS}\\
    \hline
    % \rowcolor[HTML]{C0C0C0}
    \textbf{\emph{Class} }                     & \textbf{\emph{Precision}}  & \textbf{\emph{Recall}} & \textbf{\emph{mAP@.5}}\\
    \hline
    all                                                 & 0.706          & 0.696        & 0.847         \\
    no\textunderscore helmet                            & 0.885          & 0.771        & 0.789         \\
    with\textunderscore helmet                          & 0.623          & 0.947        & 0.905         \\
    \hline
  \end{longtable}
  
  \begin{figure}[ht]
    \centering
    \includegraphics[scale=0.2]{gambar/train_v2_val/low_ligjt/customSmall/val_batch0_pred.jpg}
    % \includegraphics[scale=0.1]{gambar/train_v2_val/low_ligjt/customSmall/val_batch0_labels.jpg}
    \caption{Hasil Prediksi Pada Keadaan dengan \emph{Weight Hedec Small}}
    % \label{fig:labelbaru}  
  \end{figure}

  \item \textbf{hedec\textunderscore pure\textunderscore M }
  
  \par Dilakukan pengujian kecerahan rendah dengan menggunakan bobot yang di-\emph{train} tanpa menggunakan bobot
  pretrain COCO tetapi konfigurasi modelnya dibuat serupa dengan konfigurasi YOLOv5m. 
  Didapatkan rata - rata presisi untuk semua kelas 0.731   dan \emph{recall} untuk semuakelas 0.862 dimana lebih baik dibandingkan
  varian \emph{Small}.
  
  \begin{longtable}{|c|c|c|c|}
    \caption{Hasil Validasi Pada Tingkat Kecerahan Rendah dengan \emph{Hedec Medium}}
    \label{tb:validasitingkatacerahrendah_hedecM}\\
    \hline
    % \rowcolor[HTML]{C0C0C0}
    \textbf{\emph{Class} }                     & \textbf{\emph{Precision}}  & \textbf{\emph{Recall}} & \textbf{\emph{mAP@.5}}\\
    \hline
    all                                                 & 0.731          & 0.862       & 0.827         \\
    no\textunderscore helmet                            & 0.799          & 0.793       & 0.757         \\
    with\textunderscore helmet                          & 0.663          & 0.93        & 0.897         \\
    \hline
  \end{longtable}
  
  \begin{figure}[ht]
    \centering
    \includegraphics[scale=0.2]{gambar/train_v2_val/low_ligjt/customMedium/val_batch0_pred.jpg}
    % \includegraphics[scale=0.1]{gambar/train_v2_val/low_ligjt/customMedium/val_batch0_labels.jpg}
    \caption{Hasil Prediksi Pada Keadaan dengan \emph{Weight Hedec Medium}}
    % \label{fig:labelbaru}  
  \end{figure}

  \item \textbf{hedec\textunderscore pure\textunderscore L }
  \par Dilakukan pengujian kecerahan rendah dengan menggunakan bobot yang di-\emph{train} tanpa menggunakan bobot
  pretrain COCO tetapi konfigurasi modelnya dibuat serupa dengan konfigurasi YOLOv5l. 
  Didapatkan rata - rata presisi untuk semua kelas 0.743   dan \emph{recall} untuk semuakelas 0.755.
  \par Berbeda dengan bobot imbangan variasi \emph{Large} yang dilatih menggunakan Pretrained COCO, \emph{overall precision} pada varian ini
  sedikit lebih baik daripada varian \emph{Medium} nya, tetapi untuk \emph{Recall} sama - sama lebih kecil.
  
  \begin{longtable}{|c|c|c|c|}
    \caption{Hasil Validasi Pada Tingkat Kecerahan Rendah dengan \emph{Hedec Large}}
    \label{tb:validasitingkatacerahrendah_hedecL}\\
    \hline
    % \rowcolor[HTML]{C0C0C0}
    \textbf{\emph{Class} }                     & \textbf{\emph{Precision}}  & \textbf{\emph{Recall}} & \textbf{\emph{mAP@.5}}\\
    \hline
    all                                                 & 0.743          & 0.755       & 0.789         \\
    no\textunderscore helmet                            & 0.85           & 0.65        & 0.733         \\
    with\textunderscore helmet                          & 0.663          & 0.86        & 0.845         \\
    \hline
  \end{longtable}
  
  \begin{figure}[ht]
    \centering
    \includegraphics[scale=0.2]{gambar/train_v2_val/low_ligjt/customLarge/val_batch0_pred.jpg}
    % \includegraphics[scale=0.1]{gambar/train_v2_val/low_ligjt/customLarge/val_batch0_labels.jpg}
    \caption{Hasil Prediksi Pada Keadaan dengan \emph{Weight Hedec Large}}
    % \label{fig:labelbaru}  
  \end{figure}
  
\end{enumerate}


\subsection{Pengujian Model YOLOv5 dengan \emph{Helmet Detection Weights} Pada Jetson Nano}
\label{subsec:jetsonnano_hedectest}

\par Bagian ini merupakan pemaparan hasil pengujian performa Model YOLOv5 dengan bobot - bobot yang sudah di\emph{train}
sebelumnya pada Jetson Nano. Tujuan dari pengujian ini yaitu membandingkan effektifitas dari masing - masing
bobot yang sudah dibuat pada aspek performa \emph{inference} mengingat Jetson Nano merupakan \emph{mini-computer} yang memiliki
\emph{Graphic Processing Unit}(GPU)nya sendiri dan memang didesain untuk aplikasi AI IoT. Pengujuran yang diambil pada pengujian ini
yaitu nilai \emph{Frame Per Second}(FPS).

\par Pengujian dilakukan menggunakan Webcam Nemesis NYK A-90 Everest yang dipasang dengan Jetson Nano melalui kabel USB dengan resolusi
input 256x192. Input resolusi 256x192 yang terbilang sangat kecil ini digunakan karena resolusi diatasnya seperti 480p dan 640p akan menghasilkan
nilai FPS yang sangat kecil atau jika tidak menyebabkan terminal menghentikan secara proses \emph{inference} karena melebihi kapasitas VRAM dari Jetson Nano-nya
sehingga pengukuran performa sulit untuk dilakukan. Hasil pengukuran FPS pada pengujian YOLOv5 Helmet Detection dipaparkan pada Tabel~\ref{tb:jetsonanoperformancetest}.



\begin{longtable}{|c|c|}
  \caption{Hasil Test Peforma Pad Jetson Nano}
  \label{tb:jetsonanoperformancetest}\\
  \hline
  % \rowcolor[HTML]{C0C0C0}
  \textbf{\emph{Class} }                     & \textbf{FPS}  \\
  \hline
  hedec\textunderscore pretrain\textunderscore N                                   & 24.4          \\
  hedec\textunderscore pretrain\textunderscore S                                   & 22.2          \\
  hedec\textunderscore pretrain\textunderscore M                                   & 15.2          \\
  hedec\textunderscore pretrain\textunderscore L                                   & NULL          \\
  hedec\textunderscore pure\textunderscore N                                       & 24.9          \\
  hedec\textunderscore pure\textunderscore S                                       & 23          \\
  hedec\textunderscore pure\textunderscore M                                       & 15          \\
  hedec\textunderscore pure\textunderscore L                                       & NULL          \\
  \hline
\end{longtable}

\par Berdasarkan hasil pengujian pada ~\ref{tb:jetsonanoperformancetest}, terdapat beberapa poin analisis yang bisa diambil.
Performa dari nilai \emph{Frame Rate} dari \emph{weight} yang menggunakan \emph{Pretrained Weights} dengan pasangan bandingnya di \emph{weight}
yang tidak menggunakan \emph{Pretrained Weights} memiliki performa yang tidak berbeda jauh. Sewajarnya, varian \emph{Nano (N)} dari
\emph{Pretrained Weight} atau \emph{Pure Weight} memiliki FPS paling tinggi dibandingkan varian diatasnya yaitu S,M,dan L.
Pengujian pada varian \emph{Large (L)} selalu berakhir dengan terminal mematikan proses \emph{inference}
dikarenakan VRAM dari Jetson Nano tidak memadai proses \emph{inference} nya sendiri sehingga data yang dimasukan pada Tabel L NULL.