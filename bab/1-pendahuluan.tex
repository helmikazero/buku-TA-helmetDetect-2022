\chapter{PENDAHULUAN}
\label{chap:pendahuluan}

% Ubah bagian-bagian berikut dengan isi dari pendahuluan

\section{Latar Belakang}
\label{sec:latarbelakang}

Helm keselamatan kerja atau safety helmet merupakan salah satu Alat Pelindung Diri yang diatur dalam 
Peraturan Menteri Tenga Kerja dan Transmigrasi Republik Indonesia NOMOR PER.08/MEN/VII/2010 tentang ALAT PELINDUNG DIRI \cite{suratkementriantenagakerja}. Seperti helm pada umumnya, Helm Keselamatan Kerja berfungsi untuk melindungi kepala pengguna dari segala bentuk benturan atau hantaman langsung ke kepala penggunanya yang dimana dapat mengurangi kemungkinana cedera.

Tetapi walaupun sudah diatur dalam aturan pemerintah belum menjamin semua pekerja menggunakan Helm Keselamatan Kerja saat diinstruksikan untuk digunakan. Melalui jurnal Menakar Implementasi Kebijakan Keselamatan dan Kesehatan Kerja di Indonesia oleh Masrully pada tahun 2019, Sekretaris Umum BPD Gabungan Pelaksana Konstruksi Indonesia atau GAPENSI menyatakan bahwa menurutnya sejumlah proyek konstruksi yang digarap perusahaan BUMN sering didapati pekerja yang mengabaikan keselamatan kerja yang dimana waktu itu bahasan utamanya adalah kecelakan kerja yang marak terjadi sepanjang tahun 2017 hingga 2018 \cite{masrully2019menakar}.

Dalam penanggulangan kelalaian penggunaan helm keselamatan kerja ataupun APD lainnya, perusahaan - perusahaan yang melakukan pekerjaan pada umumnya sudah mengerahkan supervisor atau pengawasa berupa petugas K3 atau Ahli K3 yang dimana juga bertugas untuk mengawasi penggunaan APD sebagai salah satu bentuk K3. Selain itu pengerahnnya  ini sendiri pun juga diatur dalam PERATURAN MENTERI PEKERJAAN UMUM NOMOR : 05/PRT/M/2014 Pasal 6 ayat 1 dan 2  yang menyebutkan bahwa wajib melibatkankan ahli atau petugas K3 pada potensi bahaya rendah atau tinggi \cite{kementrianpekerjaanumum}. Tetapi, petugas K3 yang dikerahkankan pada umumnya masih melakukan pengawasan secara manual. Disini seperti yang diketahui yaitu manusia memilki batasan tertentu dimana luas area pengawasan yang terlalu luas dan banyaknya jumlah pekerja yang harus diawasi menjadi tantangan.

Terlepas dari adanya aturan K3 dan pengerahan pengawasan dengan segala keterbatasannya, berdasarkan Data Kecelakaan dan Penyakit Akibat Kerja Triwulan II 2020 dari Kemnaker, kecelakaan kerja Tipe A yang meliputi "Terbentur pada umumnya menunjukan kontak atau persinggungan dengan benda tajam atau benda keras yang menyebabkan tergores, terpotong, tertusuk dll" mencapai angka 878 kecelakaan dimana menjadi angka terbesar dibanding tipe kecelakaan lain dengan Tipe J (lain-lain) dengan angka 637 dan Tipe C (terjepit) dengan angka 439 \cite{satudata_kecelakaan_kerja}.


\section{Permasalahan}
\label{sec:permasalahan}

Berdasarkan latar belakang tersebut, ditarik suatu permasalahan untuk judul ini yaitu Helm keselamatan kerja sebagai salah satu APD K3 masih sering diabaikan ditambah dengan pengawasan penggunaan helm sebagai APD K3 yang masih dilakukan secara manual oleh petugas pengawas.


\section{Batasan Masalah}
\label{sec:batasanmasalah}

Dalam pengerjaan judul penelitian ini, deteksi hanya dilakukan untuk penggunaan helm keselamatan kerja.

\section{Tujuan}
\label{sec:Tujuan}

Dari permasalahan yang disebutkan, dapat dapat ditentukan tujuan : Merancang sistem yang dapat mendeteksi penggunaan helm keselamatan kerja secara real-time.

\section{Manfaat}
\label{sec:manfaat}

Manfaat dari keluaran dari penelitian ini yaitu mempermudah proses pengawasan terhadap penggunaan helm keselamatan kerja oleh pekerja.

% Format Buku TA baru, ga pake sistematika penulisan

% \section{Sistematika Penulisan}
% \label{sec:sistematikapenulisan}

% Laporan penelitian tugas akhir ini terbagi menjadi \lipsum[1][1-3] yaitu:

% \begin{enumerate}[nolistsep]

%   \item \textbf{BAB I Pendahuluan}

%   Bab ini berisi \lipsum[2][1-5]

%   \vspace{2ex}

%   \item \textbf{BAB II Tinjauan Pustaka}

%   Bab ini berisi \lipsum[3][1-5]

%   \vspace{2ex}

%   \item \textbf{BAB III Desain dan Implementasi Sistem}

%   Bab ini berisi \lipsum[4][1-5]

%   \vspace{2ex}

%   \item \textbf{BAB IV Pengujian dan Analisa}

%   Bab ini berisi \lipsum[5][1-5]

%   \vspace{2ex}

%   \item \textbf{BAB V Penutup}

%   Bab ini berisi \lipsum[6][1-5]

% \end{enumerate}
