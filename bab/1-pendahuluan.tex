\chapter{PENDAHULUAN}
\label{chap:pendahuluan}

% Ubah bagian-bagian berikut dengan isi dari pendahuluan

\section{Latar Belakang}
\label{sec:latarbelakang}

Keselamatan dan Kesehatan Kerja atau biasa disingkat K3 menjadi usaha untuk meningkatkan kualitas lingkungan kerja agar menjadi lebih aman dan sehat bagi segala pihak yang ada di lingkungan tersebut. Aman dan sehat bisa dalam artian bebas dari kecelakaan, kebakaran, ledakan, lingkungan tercemar, atau wabah penyakit. Tentu saja hal - hal tersebut perlu dihindari karena dapat memberikan dampak kerugian material dan bahkan melayangnya nyawa manusia. Aturan K3 sendiri diatur dalam bentuk norma oleh regulasi pemerintah Republik Indonesia lewat UUD 1945 pasal 27 ayat 2 tentang filosofi penghidupan yang layak, UU No 1 tahun 1970 tentang keselamatan kerja, Undang-Undang No. 13 Tahun 2003 pasal 86 dan 87 Kewajiban penerapan Sistem Manajemen Keselamatan dan Kesehatan Kerja (SMK3), Peraturan Pemerintah No.50 Tahun 2012 tentang Sistem Manajemen Keselamatan dan Kesehatan Kerja (SMK3), dan peraturan pelaksanaan lainnya dari Permenaker, Instruksi Menteri, Pekmenaker.  \cite{ahlik3umum-k3indonesia_2021}
Medan dari proyek konstruksi dapat dianggap sebagai lingkungan yang penuh dengan resiko menjadikannya suatu hal yang patut dipertimbangkan. Bukan hal yang jarang dimana para personel lapangan yang bekerja di suatu proyek konstruksi mengalami cedera akibat hal - hal tertentu. Mulai dari debu konstruksi, puing - puing berterbangan, jatuh dari ketinggian, atau tertimpa benda. Cedera kepala oleh jatuh ketinggian atau tertimpa benda dapat berakibat fatal pada nyawa personil lapangan. \cite{li2020deep}
Helm keselamatan kerja atau \emph{Hardhat} dalam Bahasa inggris atau juga kadang disebut Helm proyek merupakan salah satu bentuk APD K3 yang berfungsi untuk melindungi kepala pengguna dari benturan. Bentuk benturan contohnya kejatuhan benda tajam atau berat yang sekiranya jika tidak menggunakan pelindung akan berdampak fatal pada kepala personil proyek. Selain benturan, helm juga digunakan untuk melindungi kepala penggunanya dari percikan api dan berbagai bentuk serpihan terbang yang biasa ada di lokasi kerja.\cite{k3_mutiaramutu}
Adanya aturan penggunaan helm di suatu proyek konstruksi dengan dasar K3 belum tentu menjamin penggunaan efektif dari helm proyek tersebut. 
Berdasarkan Data Kecelakaan dan Penyakit Akibat Kerja Triwulan II 2020 dari Kemnaker, kecelakaan kerja Tipe A yang meliputi "Terbentur pada umumnya menunjukan kontak atau persinggungan dengan benda tajam atau benda keras yang menyebabkan tergores, terpotong, tertusuk dll" mencapai angka 878 kecelakaan dimana menjadi angka terbesar dibanding tipe kecelakaan lain dengan Tipe J (lain-lain) dengan angka 637 dan Tipe C (terjepit) dengan angka 439 \cite{satudata_kecelakaan_kerja}.
Pengawasan terhadap penerapan Keselamatan Kesehatan Kerja pada suatu proyek seperti menggunakan helm Hard Hat secara konvensional sudah sering dilakukan. Personil pengawas yang dikerahkan untuk memastikan para pekerja di suatu proyek mematuhi aturan keselamatan kerja. Misalnya pengawas ditugaskan untuk mengingatkan pekerja proyek yang tidak menggunakan helm proyek dengan tepat atau bahkan tidak digunakan sama sekali.\cite{li2020deep}


\section{Permasalahan}
\label{sec:permasalahan}

Berdasarkan latar belakang tersebut, ditarik suatu permasalahan untuk judul ini yaitu Helm keselamatan kerja sebagai salah satu APD K3 masih sering diabaikan ditambah dengan pengawasan penggunaan helm sebagai APD K3 yang masih dilakukan secara manual oleh petugas pengawas.


\section{Batasan Masalah}
\label{sec:batasanmasalah}

Batasan-batasan dari \lipsum[1][1-3] adalah:

\begin{enumerate}[nolistsep]

  \item Mempermudah \lipsum[2][1-3]

  \item \lipsum[3][1-5]

  \item \lipsum[4][1-5]

\end{enumerate}

\section{Tujuan}
\label{sec:Tujuan}

Dari permasalahan yang disebutkan, dapat dapat ditentukan tujuan :

\begin{enumerate}[nolistsep]

  \item Merancang sistem yang dapat mendeteksi penggunaan helm keselamatan kerja secara real-time

\end{enumerate}

\section{Manfaat}
\label{sec:manfaat}

Manfaat dari \lipsum[1][1-3] adalah:

\begin{enumerate}[nolistsep]

  \item Mempermudah \lipsum[2][1-3]

  \item \lipsum[3][1-5]

  \item \lipsum[4][1-5]

\end{enumerate}

% Format Buku TA baru, ga pake sistematika penulisan

% \section{Sistematika Penulisan}
% \label{sec:sistematikapenulisan}

% Laporan penelitian tugas akhir ini terbagi menjadi \lipsum[1][1-3] yaitu:

% \begin{enumerate}[nolistsep]

%   \item \textbf{BAB I Pendahuluan}

%   Bab ini berisi \lipsum[2][1-5]

%   \vspace{2ex}

%   \item \textbf{BAB II Tinjauan Pustaka}

%   Bab ini berisi \lipsum[3][1-5]

%   \vspace{2ex}

%   \item \textbf{BAB III Desain dan Implementasi Sistem}

%   Bab ini berisi \lipsum[4][1-5]

%   \vspace{2ex}

%   \item \textbf{BAB IV Pengujian dan Analisa}

%   Bab ini berisi \lipsum[5][1-5]

%   \vspace{2ex}

%   \item \textbf{BAB V Penutup}

%   Bab ini berisi \lipsum[6][1-5]

% \end{enumerate}
