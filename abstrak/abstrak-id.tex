\begin{center}
  \large\textbf{ABSTRAK}
\end{center}

\addcontentsline{toc}{chapter}{ABSTRAK}

\vspace{2ex}

\begingroup
  % Menghilangkan padding
  \setlength{\tabcolsep}{0pt}

  \noindent
  \begin{tabularx}{\textwidth}{l >{\centering}m{2em} X}
    % Ubah kalimat berikut dengan nama mahasiswa
    Nama Mahasiswa    &:& Helmika Mahendra Priyanto \\

    % Ubah kalimat berikut dengan judul tugas akhir
    Judul Tugas Akhir &:&	Deteksi Helm Keselamatan Kerja menggunakan CNN \\

    % Ubah kalimat-kalimat berikut dengan nama-nama dosen pembimbing
    Pembimbing        &:& 1. Reza Fuad Rachmadi, S.T., M.T., Ph.D. \\
                      & & 2. Dr.Supeno Mardi Susiki Nugroho ST., M.T. \\
  \end{tabularx}
\endgroup

% Ubah paragraf berikut dengan abstrak dari tugas akhir
Keselamatan dan Kesehatan Kerja bertujuan meningkatkan standar dan kualitas kerja di era modern ini. Pengaplikasiannya pun beragam, salah satunya yaitu penerapan penggunaan helm keselamatan kerja atau helm proyek atau Hard Hat. Medan proyek konstruksi yang berisiko tinggi menjadi alasan utama pekerja proyek harus benar - benar mematuhi aturan penggunaan APD, dimana salah satunya penggunaan helm proyek. Helm proyek membantu mengurangi dampak benturan misal saat pengguna terjatuh atau tertimpa benda berat atau tajam. Tetapi tidak semua personel lapangan akan dengan sendirinya mematuhi aturan ini sehingga diperlukannya pengawasan dalam penerapan penggunaan helm proyek sebagai salah satu APD keselamatan kerja. Pengawas atau supervisor lapangan yang dikerahkan adalah personil manusia yang juga memiliki batasannya sebagai manusia. Kondisi lapangan proyek yang luas dan banyaknya personil lapangan akan menjadi suatu kesulitan untuk pengawas manusia untuk memastikan tiap personil lapangan mematuhi aturan penggunaan helm keselamatan kerja. Maka dari itu, dalam penelitian ini diambil suatu tujuan yaitu merancang sistem yang dapat mendeteksi penggunaan helm proyek secara otomatis. Dalam perancangan sistem ini, akan memanfaat Convolutional Neural Network yang didesain untuk rekognisi data dua dimensi. Sistem yang sudah jadi akan diuji pada lapangan proyek konstruksi.


% Ubah kata-kata berikut dengan kata kunci dari tugas akhir
Kata Kunci: Visi Komputer, \emph{YouV Only Look Once} (YOLO), Helm Keselamatan Kerja
