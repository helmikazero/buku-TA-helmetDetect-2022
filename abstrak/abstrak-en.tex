\begin{center}
  \large\textbf{ABSTRACT}
\end{center}

\addcontentsline{toc}{chapter}{ABSTRACT}

\vspace{2ex}

\begingroup
  % Menghilangkan padding
  \setlength{\tabcolsep}{0pt}

  \noindent
  \begin{tabularx}{\textwidth}{l >{\centering}m{3em} X}
    % Ubah kalimat berikut dengan nama mahasiswa
    \emph{Name}     &:& Helmika Mahendra Priyanto \\

    % Ubah kalimat berikut dengan judul tugas akhir dalam Bahasa Inggris
    \emph{Title}    &:& \emph{Safety Helmet Detection Using CNN} \\

    % Ubah kalimat-kalimat berikut dengan nama-nama dosen pembimbing
    \emph{Advisors} &:& 1. Reza Fuad Rachmadi ST., MT., Ph.D. \\
                    & & 2. Dr.Supeno Mardi Susiki Nugroho ST., M.T. \\
  \end{tabularx}
\endgroup

% Ubah paragraf berikut dengan abstrak dari tugas akhir dalam Bahasa Inggris
\emph{
  Occupational Health and Safety aim to improve the standard and quality of the work environment in the modern era. There are several ways on implementing Occupational Health and Safety in which the use of a Safety Helmet or Hard hat is one of them. The fact that the construction site is considered to be a high-risk work environment is the reason why obeying the strict rules of wearing the safety helmet is important. A hard hat or safety helmet helps by reducing the effect of the impact caused by heavy or sharp objects falling directly to the user’s head. But even with the daunting fact of a possible fatality caused by head injuries caused by falling objects, construction workers sometimes will not wear the safety helmet of their own volition. Therefore, supervision is needed to ensure every worker wears the safety helmet as instructed. Conventional methods have been used to supervise the use of safety helmets by deploying human supervisors which have their own disadvantages. The vast area of the construction sites and the number of personnel that is more than a human can count is a challenge for a human supervisor to carry on their duty to supervise every personnel in the area. Therefore, this research aims to develop a system that is capable of automatically detecting the personnel that wears safety helmets and the ones that do not wear a safety helmet and triggered sort of alarm when the system detects personnel that doesn’t wear a safety helmet properly. The development of the system will be utilizing Convolutional Neural Network that is mainly designed to do 2D recognition. }

% Ubah kata-kata berikut dengan kata kunci dari tugas akhir dalam Bahasa Inggris
\emph{Keywords}: \emph{Rocket}, \emph{Anti-gravity}, \emph{Energy}, \emph{Space}.
