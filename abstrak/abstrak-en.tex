\begin{center}
  \large\textbf{ABSTRACT}
\end{center}

\addcontentsline{toc}{chapter}{ABSTRACT}

\vspace{2ex}

\begingroup
  % Menghilangkan padding
  \setlength{\tabcolsep}{0pt}

  \noindent
  \begin{tabularx}{\textwidth}{l >{\centering}m{3em} X}
    % Ubah kalimat berikut dengan nama mahasiswa
    \emph{Name}     &:& Helmika Mahendra Priyanto \\

    % Ubah kalimat berikut dengan judul tugas akhir dalam Bahasa Inggris
    \emph{Title}    &:& \emph{Safety Helmet Detection Using CNN} \\

    % Ubah kalimat-kalimat berikut dengan nama-nama dosen pembimbing
    \emph{Advisors} &:& 1. Reza Fuad Rachmadi ST., MT., Ph.D. \\
                    & & 2. Dr.Supeno Mardi Susiki Nugroho ST., M.T. \\
  \end{tabularx}
\endgroup

% Ubah paragraf berikut dengan abstrak dari tugas akhir dalam Bahasa Inggris
\par \emph{
  Safety Helmet or Hardhat is one of the Personal Protective Equipment which purpose is to protect the wearer from direct impact
  to the head. There are regulations that stated the importance and obligated the use of Personal Protective Equipment in which 
  Safety Helmet is included such as Regulation of Republic of Indonesia Ministry of Manpower \emph{NOMOR PER.08/MEN/VII/2010} for
  PERSONAL PROTECTIVE EQUIPMENT REGULATION. Despite the obligation of the use of safety helmet as one of the Personal Protective Equipment,
  it does not guarantee all the field workers will wear the helmet. Companies that held the constructions usually had already deployed
  supervisor to ensure that all worker wear the safety helmet. The deployment of the supervisor itself is also regulated in one of the regulations from
  Republic of Indonesia Ministry of Manpower. But the method that is used to do supervision
  is still done manually by the supervisor which has its limitations. The vast area of the 
  construction sites and the number of personnel that is more than a 
  human can count is a challenge for a human supervisor to carry on 
  their duty to supervise every personnel in the area.
  Therefore, this research aims to develop a system that is capable of 
  automatically detecting the personnel that wears safety helmets and the 
  ones that do not wear a safety helmet and triggering a sort of alarm when the system 
  detects personnel that does not wear a safety helmet properly. 
  The development of the system will be utilizing a Convolutional Neural Network, mainly designed to do 2D recognition.
}


% Ubah kata-kata berikut dengan kata kunci dari tugas akhir dalam Bahasa Inggris
\emph{Keywords}: \emph{Computer Vision}, \emph{You Only Look Once} YOLO, \emph{Safety Helmet}.
